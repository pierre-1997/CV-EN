%%%%%%%%%%%%%%%%%
% This is an sample CV template created using altacv.cls
% (v1.7, 9 August 2023) written by LianTze Lim (liantze@gmail.com). Compiles with pdfLaTeX, XeLaTeX and LuaLaTeX.
%
%% It may be distributed and/or modified under the
%% conditions of the LaTeX Project Public License, either version 1.3
%% of this license or (at your option) any later version.
%% The latest version of this license is in
%%    http://www.latex-project.org/lppl.txt
%% and version 1.3 or later is part of all distributions of LaTeX
%% version 2003/12/01 or later.
%%%%%%%%%%%%%%%%

%% Use the "normalphoto" option if you want a normal photo instead of cropped to a circle
% \documentclass[10pt,a4paper,normalphoto]{altacv}

\documentclass[10pt,a4paper,ragged2e,withhyper]{altacv}
%% AltaCV uses the fontawesome5 and packages.
%% See http://texdoc.net/pkg/fontawesome5 for full list of symbols.

% Change the page layout if you need to
\geometry{left=1.25cm,right=1.25cm,top=1cm,bottom=1cm,columnsep=1.2cm}

% The paracol package lets you typeset columns of text in parallel
\usepackage{paracol}

% Change the font if you want to, depending on whether
% you're using pdflatex or xelatex/lualatex
% WHEN COMPILING WITH XELATEX PLEASE USE
% xelatex -shell-escape -output-driver="xdvipdfmx -z 0" sample.tex
\iftutex
  % If using xelatex or lualatex:
  \setmainfont{Roboto Slab}
  \setsansfont{Lato}
  \renewcommand{\familydefault}{\sfdefault}
\else
  % If using pdflatex:
  \usepackage[rm]{roboto}
  \usepackage[defaultsans]{lato}
  % \usepackage{sourcesanspro}
  \renewcommand{\familydefault}{\sfdefault}
\fi

% Change the colours if you want to
\definecolor{SlateGrey}{HTML}{2E2E2E}
\definecolor{LightGrey}{HTML}{666666}
\definecolor{DarkPastelRed}{HTML}{450808}
\definecolor{PastelRed}{HTML}{8F0D0D}
\definecolor{GoldenEarth}{HTML}{E7D192}
\colorlet{name}{black}
\colorlet{tagline}{PastelRed}
\colorlet{heading}{DarkPastelRed}
\colorlet{headingrule}{GoldenEarth}
\colorlet{subheading}{PastelRed}
\colorlet{accent}{PastelRed}
\colorlet{emphasis}{SlateGrey}
\colorlet{body}{LightGrey}

% Change some fonts, if necessary
\renewcommand{\namefont}{\Huge\rmfamily\bfseries}
\renewcommand{\personalinfofont}{\footnotesize}
\renewcommand{\cvsectionfont}{\LARGE\rmfamily\bfseries}
\renewcommand{\cvsubsectionfont}{\large\bfseries}


% Change the bullets for itemize and rating marker
% for \cvskill if you want to
\renewcommand{\cvItemMarker}{{\small\textbullet}}
\renewcommand{\cvRatingMarker}{\faCircle}
% ...and the markers for the date/location for \cvevent
% \renewcommand{\cvDateMarker}{\faCalendar*[regular]}
% \renewcommand{\cvLocationMarker}{\faMapMarker*}


% If your CV/résumé is in a language other than English,
% then you probably want to change these so that when you
% copy-paste from the PDF or run pdftotext, the location
% and date marker icons for \cvevent will paste as correct
% translations. For example Spanish:
% \renewcommand{\locationname}{Ubicación}
% \renewcommand{\datename}{Fecha}


%% Use (and optionally edit if necessary) this .tex if you
%% want to use an author-year reference style like APA(6)
%% for your publication list
% \input{pubs-authoryear.cfg}

%% Use (and optionally edit if necessary) this .tex if you
%% want an originally numerical reference style like IEEE
%% for your publication list
\input{pubs-num.cfg}

%% sample.bib contains your publications
\addbibresource{sample.bib}

\begin{document}
\name{Pierre KETOBIAKOU}
\tagline{Rust Developer with Expertise in Cybersecurity}
%% You can add multiple photos on the left or right
% \photoR{2.8cm}{me}
% \photoL{2.5cm}{Yacht_High,Suitcase_High}

\personalinfo{%
  % Not all of these are required!

  % \email{ketobiakou@et.esiea.fr}
  \email{pierre.097@proton.me}
  \phone{+33 (0)7 82 57 44 62}
  % \mailaddress{2 bis route des Varennes}
  \location{PARIS, France}
  \github{pierre-1997}

  % \homepage{www.homepage.com}
  % \twitter{@twitterhandle}
  % \linkedin{your_id}
  % \orcid{0000-0000-0000-0000}
  %% You can add your own arbitrary detail with
  %% \printinfo{symbol}{detail}[optional hyperlink prefix]
  % \printinfo{\faPaw}{Hey ho!}[https://example.com/]
  %% Or you can declare your own field with
  %% \NewInfoFiled{fieldname}{symbol}[optional hyperlink prefix] and use it:
  % \NewInfoField{gitlab}{\faGitlab}[https://gitlab.com/]
  % \gitlab{your_id}
  %%
  %% For services and platforms like Mastodon where there isn't a
  %% straightforward relation between the user ID/nickname and the hyperlink,
  %% you can use \printinfo directly e.g.
  % \printinfo{\faMastodon}{@username@instace}[https://instance.url/@username]
  %% But if you absolutely want to create new dedicated info fields for
  %% such platforms, then use \NewInfoField* with a star:
  % \NewInfoField*{mastodon}{\faMastodon}
  %% then you can use \mastodon, with TWO arguments where the 2nd argument is
  %% the full hyperlink.
  % \mastodon{@username@instance}{https://instance.url/@username}
}

\makecvheader
%% Depending on your tastes, you may want to make fonts of itemize environments slightly smaller
\AtBeginEnvironment{itemize}{\small}

%% Set the left/right column width ratio to 6:4.
\columnratio{0.5}

% Start a 2-column paracol. Both the left and right columns will automatically
% break across pages if things get too long.
\begin{paracol}{2}
\cvsection{Experience}

\cvevent{Backend Developer}{Blackfoot}{April 2023 -- July 2024}{PARIS, France}
\begin{itemize}
  \item Achieved Rust proficiency through designing multi-server backend systems, enhancing application security and performance.
  \item Led the design and implementation of scalable, secure backend architectures using Rust (actix, tokio, ...) for improved performance.
  \item Technical stack: kubernetes, GCP, Rust, Linux, Docker, Git, PostgreSQL/MongoDB
\end{itemize}

\divider

\cvevent{SOC Architect}{Orange CyberDefense}{March 2021 -- March 2022}{PARIS, France}
\begin{itemize}
  \item Orchestrated the development of a SOC for large-scale enterprises, integrating advanced automation to enhance security monitoring and response efficiency. % in collaboration with international teams.
  % \item Deep understanding of SOC as a service for big companies
  \item Technical stack: Python, Cortex XSOAR, Docker, Gitlab
\end{itemize}

\divider

\cvevent{SOC Developer}{Engie}{September 2019 -- August 2020}{SAINT-OUEN, France}
\begin{itemize}
  \item Developed phishing detection capabilities for enterprise-level security systems, contributing to the overall SOC architecture.
  \item Technical stack: Python, Resilient (IBM), Jira, Confluence, Docker
\end{itemize}

\divider

\cvevent{Assistant Web Developer}{IDVroom}{July to August 2018}{ANGERS, France}
\begin{itemize}
  \item Contributed to the internationalization (i18n) of the website, gaining hands-on experience with the Model View Controller (MVC) architecture
  % \item Worked on the Internationalization (i18n) of their website
  % \item Learned a lot about the Model View Controller architecture
  \item Technical stack: Zend, Jira, Scrum, Linux
\end{itemize}

% \divider
% \cvevent{Team Member}{Quick / McDonald's}{Summer 2016 / Summer 2017}{St Sylvain d'Anjou, France}
% \begin{itemize}
%   \item Prepared and cooked customer orders
%   \item Discovered basics of the service industry
% \end{itemize}
%
\cvsection{Education}

\cvevent{Engineering School}{ESIEA, LAVAL}{Sept 2015 -- Sept 2020}{}
\begin{itemize}
 \item Information System domain, Cybersecurity major
 \item Obtained with honors
\end{itemize}

\divider

\cvevent{High school French scientific baccalaureate}{Lycée St Aubin la Salle}{June 2015}{}
\begin{itemize}
\item Mathematics specialty
\item Obtained with honors
\end{itemize}

% use ONLY \newpage if you want to force a page break for
% ONLY the current column
% \newpage

%% Switch to the right column. This will now automatically move to the second
%% page if the content is too long.
\switchcolumn

\cvsection{Projects}

\cvevent{Rust Cryptography and System Projects}{Personal project}{December 2021 -- April 2023}{Home}
\begin{itemize}
  \item Self-taught Rust development, with a focus on cryptography, algorithm design, and system performance, to secure a position in software engineering.
  % \item Learning Rust on my own to find a position as developer
  % \item Projects: Cryptopals, Crafting Interpreters, Compression algorithms, AoC, Sudoku, Proxy, ...
  \item Technical stack: Rust, Cargo, Cryptography papers
\end{itemize}

\divider

\cvevent{Learning Offensive Cybersecurity}{Personal project}{September 2020 -- February 2021}{Home}
\begin{itemize}
  \item Learning pentesting on my own via HackTheBox to find a position in the cybersecurity industry
  \item Techniques: Web attacks, Cryptanalysis, Forensics, Reverse Engineering, Binary Exploitation
  % \item Technical stack: Archlinux VM + blackarch, cutter, postman
\end{itemize}

\divider

\cvevent{Blockchain based DBMS}{Indian Statistical Institute}{April to August 2019}{KOLKATA, India}
\begin{itemize}
  \item Research project: implement a blockchain-based database
  \item Discovered smart contracts and decentralized applications (DApps)
  \item Technical stack: Ethereum, MariaDB, docker, Golang
\end{itemize}

% \divider
%
% \cvevent{Security Operation Center (SOC)}{ESIEA}{4th year project}{LAVAL, France}
% \begin{itemize}
%   \item Setting up an open-source SOC for medium-sized companies
%   \item This SOC will report cyber threats/attacks to my school
% \end{itemize}

\divider

\cvevent{Cryptographic S-Boxes Analysis}{ESIEA}{3rd year project}{LAVAL, France}
\begin{itemize}
  \item Implemented a C/C++ library to evaluate and represent S-Boxes
  \item Consolidated my cryptography-related algebra knowledge (Linear/Differential matrices, Algebraic Normal Form, ...)
  \item Technical stack: C, C++, Linux
\end{itemize}

\divider

\cvevent{One-dimensional Cubli}{ESIEA}{2nd year project}{LAVAL, France}
\begin{itemize}
  \item Reproduced a one-dimensional Cubli
  \item Reinforced my choice of computer science over electronics
  \item Technical stack: C, Arduino, Solidworks
\end{itemize}


\medskip

% \cvsection{A Day of My Life}
%
% % Adapted from @Jake's answer from http://tex.stackexchange.com/a/82729/226
% % \wheelchart{outer radius}{inner radius}{
% % comma-separated list of value/text width/color/detail}
% \wheelchart{1.5cm}{0.5cm}{%
%   6/8em/accent!30/{Sleep,\\beautiful sleep},
%   3/8em/accent!40/Hopeful novelist by night,
%   8/8em/accent!60/Daytime job,
%   2/10em/accent/Sports and relaxation,
%   5/6em/accent!20/Spending time with family
% }


% \cvsection{Publications}
%
% %% Specify your last name(s) and first name(s) as given in the .bib to automatically bold your own name in the publications list.
% %% One caveat: You need to write \bibnamedelima where there's a space in your name for this to work properly; or write \bibnamedelimi if you use initials in the .bib
% %% You can specify multiple names, especially if you have changed your name or if you need to highlight multiple authors.
% \mynames{Lim/Lian\bibnamedelima Tze,
%   Wong/Lian\bibnamedelima Tze,
%   Lim/Tracy,
%   Lim/L.\bibnamedelimi T.}
% %% MAKE SURE THERE IS NO SPACE AFTER THE FINAL NAME IN YOUR \mynames LIST
%
% \nocite{*}
%
% \printbibliography[heading=pubtype,title={\printinfo{\faBook}{Books}},type=book]
%
% \divider
%
% \printbibliography[heading=pubtype,title={\printinfo{\faFile*[regular]}{Journal Articles}},type=article]
%
% \divider
%
% \printbibliography[heading=pubtype,title={\printinfo{\faUsers}{Conference Proceedings}},type=inproceedings]


% \cvsection{My Life Philosophy}
%
% \begin{quote}
% ``Something smart or heartfelt, preferably in one sentence.''
% \end{quote}
%
% \cvsection{Most Proud of}
%
% \cvachievement{\faTrophy}{Fantastic Achievement}{and some details about it}
%
% \divider
%
% \cvachievement{\faHeartbeat}{Another achievement}{more details about it of course}
%
% \divider
%
% \cvachievement{\faHeartbeat}{Another achievement}{more details about it of course}
%
% \cvsection{Strengths}
%
% \cvtag{Hard-working}
% \cvtag{Eye for detail}\\
% \cvtag{Motivator \& Leader}
%
% \divider\smallskip
%
% \cvtag{C++}
% \cvtag{Embedded Systems}\\
% \cvtag{Statistical Analysis}

\cvsection{Languages}

\cvskill{French}{5}
\divider

\cvskill{English}{4.5}
\divider

\cvskill{Spanish}{2.5} %% Supports X.5 values.

%% Yeah I didn't spend too much time making all the
%% spacing consistent... sorry. Use \smallskip, \medskip,
%% \bigskip, \vspace etc to make adjustments.
% \medskip


% \clearpage
% \divider

% \cvsection{Referees}
%
% % \cvref{name}{email}{mailing address}
% \cvref{Prof.\ Alpha Beta}{Institute}{a.beta@university.edu}
% {Address Line 1\\Address line 2}
%
% \divider
%
% \cvref{Prof.\ Gamma Delta}{Institute}{g.delta@university.edu}
% {Address Line 1\\Address line 2}


\end{paracol}


\end{document}
